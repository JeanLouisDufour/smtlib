\documentclass[a4paper]{article}

\bibliographystyle{plainurl}
\title{Computational Complexity and Statistical Physics}
\author{Jean-Louis Dufour}

\usepackage{graphicx}
\graphicspath{{./lectures_img/}}
\usepackage{hyperref}
%\usepackage{natbib}
%\usepackage{tikz}

\begin{document}

%\nocite{*}
\maketitle

\begin{abstract}
History of the links between Computational Complexity and Statistical Physics
\end{abstract}

\section{Introduction to computational complexity} \label{section:Introduction}

\paragraph{What is SAT ?}





\section{Foo}

\paragraph*{Intro}

The first to link neurophysiology and logic are \cite{mcculloch1943logical}

\begin{figure}[h]
  \centering
  \begin{minipage}[b]{0.55\textwidth}
    \includegraphics[width=\textwidth]{McCulloch.png}
    \caption{Warren McCulloch}
  \end{minipage}
  \hfill
  \begin{minipage}[b]{0.4\textwidth}
    \includegraphics[width=\textwidth]{Pitts.png}
    \caption{Walter Pitts}
  \end{minipage}
\end{figure}

bla

\begin{figure}[h]
\includegraphics[width=1\textwidth]{43_mcculloch_pitts_0.png}
\caption{Warren McCulloch}
\end{figure}

bla

\begin{figure}[h]
\includegraphics[width=1\textwidth]{43_mcculloch_pitts_1.png}
\caption{Warren McCulloch}
\end{figure}

\begin{figure}[h]
\includegraphics[width=1\textwidth]{43_mcculloch_pitts_2.png}
\caption{Warren McCulloch}
\end{figure}

Real-time scheduling is more complex on a multi-core processor than on a single-core processor, from two points of view:
\begin{description}
\item[Task duration]
estimation (WCET) is more complex, because of the interference between the cores on the shared bus,
\item[Task ordering]
is more complex, because of the combinatorial explosion of the number of possibilities.
\end{description}
These two points of view are not orthogonal: ordering impacts interference.
But in an industrial context, to keep things manageable we have no other choice (given the State of the Art \cite{DBLP:journals/csur/MaizaRRGAD19})
than to have WCETs overestimated enough to be compatible with any ordering.
In this paper we will suppose that WCETs have been safely estimated elsewhere,
and \textbf{we focus only on ordering}.
In the same way, when we speak of inter-task communications, we are only interested in the induced precedence constraints
and not in the induced CPU time.

Warning: we must precise that we address here only safety-critical systems (typically DAL A for avionics),
so our scheduling policy will look (and must be) completely trivial
compared to the scheduling policies usually dealt with: Rate-Monotonic (RM) task ordering, harmonic periods and completely static (deterministic and offline determined)
inter-task communications (no semaphore, no lock, no transactional memory).

\paragraph*{Multi-core is more complex than single-core. Why ?}

Having several cores is the root cause, but is not the explanation.

In the single-core case, task ordering has never been a subject (at least for us).
The RM policy on an harmonic set of periods is a first filter on the set of acceptable orderings.
Then come the data-flow precedence contraints, and the remaining orderings are not that numerous.
Even on our most complex applications (18 tasks on 4 periods, roughly 1ms/10ms/100ms/1s), engineering judgment does the job.

The multi-core case is a different story, and this is visible as soon as you begin to work on it, more precisely as soon as the following phenomenon is understood:
\textbf{a parallel execution of RM schedules is not necessarily RM},
because slow tasks can start before fast tasks have finished.
This sentence relies on a specific definition of `RM' in the multi-core context, it will be precised in the multi-core section.
Anyway, the solution is easy: these slow tasks must be delayed by offsets.
But these new offsets (we don't need them in single-core schedules) add new dimensions to the problem,
and engineering judgement no more does the job.

\paragraph*{The way to an unexpected solution.}

Publications accumulate on the successes of SMT-solvers on more and more impressive Constraint Satisfaction Problems (CSP),
and it is confirmed internally on two `real-life' use-cases (FPGA verification and software test generation).
So, we have decided to automate the inference of priorities, affinities (associations cores-tasks) and offsets.
This automatisation begins with the formalization of the problem to solve, and especially two things:
\begin{itemize}
\item what is a data-flow precedence constraint ?
\item what is a multi-core rate-monotonic task schedule ?
\end{itemize}
This being done, the question `is there a task schedule satisifying the precedence constraints ?' can itself be formalized as a CSP (expressed in the SMTLIB language),
and we have written a prototype tool to generate this CSP.

The first lesson learned is that this CSP is complex (several hundred definitions and assertions for simple configurations), the generator is also a bit complex,
the CSP generated is sometimes buggy and it is not obvious at all to see when and why.
In an industrial context this is a real problem (not to mention the certification aspects), and to address it we went through the following successive stages:
\begin{enumerate}
\item
Debugging and verification mandate to develop also a simulator, that is to say a faithful model of the scheduler.
\item
The next idea is, instead of waiting passively for the bug, to tackle the problem at its source and to rearrange the CSP generator so that it mimics the simulator.
Engineers talk about `correct by construction'.
A more informed way of seeing this is to say that the CSP generator is an abstract interpretation of the simulator:
\begin{itemize}
\item the simulator emits commutation decisions and communication decisions,
\item the CSP generator emits the definitions and assertions corresponding strictly to these decisions.
\end{itemize}
\item
The second last idea is to stop spending all our time fixing the generator, but instead to directly model-check the simulator (of course slightly adapted for this purpose).
The problem is that we have used a standard scripting language (Python), and to our knowledge, there is no model-checker on it.
\item
It turns out that for application development, we use a language which comes with its model-checker: Scade (the commercial version of Lustre).
The last idea is to realize that Scade is expressive enough to implement the simulator.
\end{enumerate}
In fact, Scade capabilities in terms of array polymorphism and iterators have been decisive in order to write a simple and therefore analyzable model.

Each of these 4 steps taken separately is basic engineering. Taken together, they bring significant added value.
On the other hand, we don't claim that this approach scales well on more sophisticated scheduling policies.
Schedulers have already been model-checked \cite{DBLP:journals/stvr/Choi14}, but with a standard verification goal.
To our knowledge it is the first time that
\begin{enumerate}
\item a (model of a) scheduler is model-checked to obtain its configuration,
\item Scade (acronym for ``Safety-Critical \emph{Application} Development Environment'') is used for such a purpose.
\end{enumerate}

\paragraph*{Today's menu.}

The first section recalls the relevant aspects of our single-core policy: it was already done in \cite{destelle:hal-02267646},
we add several examples of independent but similar developments.

The second section describes our extension to the multi-core case.

The third section describes the Scade model and arguments its fidelity to the real scheduler.

The last section describes a few results obtained.

%%%%%%%%%%%%%%%%%%%%%%%%%%%%%%%%%%%%%%%%%%%%%%%%%%%%%%%%%%%%%%%%%

\section{The single-core case} \label{section:single-core}

\paragraph*{A not-so-well-known communication policy.}

As explained in \cite{destelle:hal-02267646}, we have traditionally used for our Inertial Navigation Systems a very simple single-core
multi-tasking architecture: a purely periodic preemptive scheduling (no aperiodic task: asynchronous
events are polled, with hardware time-tagging of events when needed), with a Fixed-Priority Rate-Monotonic policy.
Among the ‘standard’ portfolio of scheduling policies, we have selected 3 options:
\begin{description}
\item[Harmonic period set:]
for each pair of periods $T_i, T_j$, either $T_i$ divides $T_j$ or vice-versa. It
implies for example that the hyper-period is the longest period, and more generally, many of the
properties/bounds become simpler. Typically, the scheduling condition becomes
$\Sigma_i \frac{C_i}{T_i} \leq 1$
($C_i$ is the WCET of task $i$; this sum and each of its terms are called a ‘utilization factor’ or a ‘CPU load’),
so RM scheduling becomes an optimal policy. This is the ‘temporal’ point of view, but there is also a
‘frequential’ point of view: there are \textbf{no beats}, which simplifies greatly the functional understanding
and minimizes the number of configurations to be covered in functional test or in WCET estimation.
\item[ No ‘offset’:]
for example, at each hyper-period, all the tasks become simultaneously eligible
(and, thanks to rate-monotonicity, one having the base period is elected).
It is well known that in such ‘congestion’ situations, offsets can be very useful:
\cite{grenier:insu-02270103} gives an amazing illustration on the CAN bus,
and closer to us, if you have recently bought a ticket for the Louvre museum, you know that you have to respect the access hour mentioned.
But despite all that, in our single-core context offsets are totally useless.
\item[Constant execution time:]
tasks end with an endless loop, which is itself interrupted by the scheduler at the WCET.
We call this ‘deterministic’ scheduling, because the
number of preemptions of a task become fixed and known (offline). This is very useful when the
cache-related preemption costs (the write-backs) become significant; it was the subject of \cite{destelle:hal-02267646}.
\end{description}

A non-standard feature is ‘synchronous’ communication between tasks, which
avoids semaphores, priority inversion and all that stuff. We call it ‘synchronous’ because it has a natural model
with conditional activation in Scade \cite{camus:insu-02270095} and ‘rate transition’ blocks in Simulink \cite{DBLP:conf/rtss/MaticH05}.
The principle is incredibly simple:

\textbf{a fast task communicates with a slow task (simply by reading/writing a shared area) 
only during its first execution in the slow task period.}

The RM property applied on an harmonic period set is a sufficient condition (on a sequential execution) to guarantee that during communication,
the fast task will not be preempting the slow one.
The following figure illustrates this with a fast orange task `T1' and a (3 times) slower blue task `T2' (which is preempted one time).
The slow task communicates (reads and writes the shared area) systematically.
On the other hand, the fast task communicates only when authorized by the predicate `First(T2)',
which is True only during the first execution of T1 in the slow period of T2.
Observe that the 3rd execution of T1 doesn't preempt T2 (and we know this statically), but it doesn't communicate: it still uses 'old' T2 outputs.
The first T1 execution to benefit from `fresh' T2 outputs will be the 4th execution. 
This is a `max-latency' behavior, as opposed to `best-effort'.
%\begin{tikzpicture}
%\draw (0,0) rectangle +(2,1) (1,0.5 ) node {Fast 1};
%\draw (3,0) rectangle +(2,1) ++(1,0.5 ) node {Fast 2};
%\end{tikzpicture}
%\begin{figure}[h]
%\includegraphics[width=1\textwidth]{DMS1}
%\caption{Predefined communication times}
%\end{figure}

\paragraph*{Similar communication policies still exist.}

This principle (or the more general concept of
deterministic communication) is seldom developed in the literature: usually tasks don’t explicitly communicate, they are
 independent, but sometimes their model incorporates ‘blocking times’
corresponding to accesses to shared resources \cite{DBLP:journals/csur/DavisB11}.
We have identified three exceptions (no exhaustiveness claim) which emphasize our very same notion of deterministic/logical communication:
\begin{enumerate}
\item
the LET (Logical Execution Time) time model of the Giotto language \cite{DBLP:conf/birthday/KirschS12},
which is more general, because it just mandates that the period of a
‘communicator’ divides the periods of its readers/writers.
\item
deterministic communication is also achieved
in a still more general setting (with deterministic queueing) by the OASIS scheduler of CEA \cite{DBLP:conf/pdcs/ChabrolDALD05} and its
commercial successor ASTERIOS of Krono-Safe %\cite{krono-safe}.
\item
\cite{DBLP:journals/lites/CarlePSL15} uses a DAG task model in an ARINC
653 – like architecture with references to Giotto.
\end{enumerate}
In the four cases (‘synchronous’, LET,
OASIS and ‘ARINC653-like’), the inter-task data-flow is a key input.
Last but not least, deterministic communication permits to have identical functional
behavior between simulation on host (without preemptions) and execution on target (with preemptions).
This point is crucial and has to stay in a multi-core extension.

%%%%%%%%%%%%%%%%%%%%%%%%%%%%%%%%%%%%%%%%%%%%%%%%%%%%%%%

\section{The multi-core case} \label{section:multi-core}

\paragraph*{What is a multi-core rate-monotonic task schedule ?}

In the introduction, we said that
\textbf{a parallel execution of RM schedules is not necessarily RM},
because slow tasks can start before fast tasks have finished.
In fact it's a fuzzy statement, because the concept of `RM' is not standardized in the multi-core case.

In the single-core case, it is crystal-clear since the seminal paper of Liu and Layland \cite{DBLP:journals/jacm/LiuL73}.
They define three imbricated concepts:
\emph{priority driven} scheduling, then \emph{fixed (or static) priority} scheduling, and finally \emph{rate-monotonic} scheduling.
A \emph{rate-monotonic priority assignment} is such that
\begin{quote}
``tasks with higher request rates will have higher priorities'' (sic)
\end{quote}
A consequence in the harmonic case is that
\begin{quote}
tasks with higher request rates will run before tasks with lower request rates.
\end{quote}
(a rigourous statement would first define the meaning of `run before' between tasks of different periodicity).
In an engineering context, there is often a confusion between these two statements,
and it is not a problem as long as the application is single-core.
In fact, this confusion is very natural and convenient because it allows the engineer not to have to specify
the temporal characteristics of the data-flows between the two tasks: they are imposed by the scheduling.

In the multi-core case, the concept of `RM schedule' has no universally accepted definition.
We will assume that a multi-core schedule is simply a parallel execution of single-core schedules.
This assumption is not only technically reasonable, it opens the way to the reuse of already validated and certified single-core schedulers.
Now the question is: are the tasks independent ?
Let's illustrate this by adapting the example given in figure~1 on a 2-core processor, T1 on core 1 and T2 on core 2.
\begin{description}
\item[The first extreme interpretation]
of `multi-core RM', the fully independent case, gives the following schedule:





From a CPU time point of view, it is the most interesting option, but we must recognize that in practice, it is unrealistic,
because in applications, tasks communicate.
\item[The other extreme interpretation]
of `multi-core RM', the fully dependent case (systematic application of the principle `fast runs before slow'), gives:
%\begin{figure}[h]
%\includegraphics[width=1\textwidth]{Pitts.png}
%\caption{Walter Pitts}
%\end{figure}

\begin{figure}[h]
  \centering
  \begin{minipage}[b]{0.45\textwidth}
    \includegraphics[width=\textwidth]{Solomonoff.png}
    \caption{Solomonoff}
  \end{minipage}
  \hfill
  \begin{minipage}[b]{0.45\textwidth}
    \includegraphics[width=\textwidth]{Rapoport.png}
    \caption{Rapoport}
  \end{minipage}
\end{figure}

bla

\begin{figure}[h]
  \centering
  \begin{minipage}[b]{0.45\textwidth}
    \includegraphics[width=\textwidth]{Erdos.png}
    \caption{Erdos}
  \end{minipage}
  \hfill
  \begin{minipage}[b]{0.45\textwidth}
    \includegraphics[width=\textwidth]{Renyi.png}
    \caption{Renyi}
  \end{minipage}
 \hfill
  \begin{minipage}[b]{0.45\textwidth}
    \includegraphics[width=\textwidth]{Gilbert.png}
    \caption{Gilbert}
  \end{minipage}
\end{figure}




and the gain is of course less interesting.
But at least, this example shows clearly that T2 has to be delayed by an offset (in this case, equal to T1 duration).
In practice, either the schedulers have this capability, or you add fake tasks to simulate offsets.
\item[Our interpretation]
We have chosen the following intermediate principle:
\begin{quote}
bla
\end{quote}
 \end{description}

The only point to emphasize is that now,
tasks must have offsets: 

The solution quickly proposed, delaying the concerned slow tasks, is not optimal:
it can be the case that a slow task overlapping with a fast task, doesn't in fact communicate with this fast task.
It this case, there is absolutely no reason to delay it.




\paragraph*{What is a data-flow precedence constraint ?}

The first claim of the paper is that, if you are able to run your single-core schedulers on the multi-core platform
in a strictly synchronous way (ideally, there share the same real-time tick), with harmonic periods, a multi-core
schedule is simply a parallel combination of single-core schedules.  The
interest appears when certification is needed, because to your usual certification process, you just have to add the
proof that the parallel combination gives the intended behavior (again, this covers only the ordering aspect).





To be complete, in fact models of communication (beyond the ‘blocking time’) have also appeared to study realtime
scheduling on GPUs \cite{DBLP:conf/europar/CanonMSV18}: scheduling tasks on a highly-parallel architecture without taking into
account communications explicitly is just nonsense. The set of tasks is in fact a graph (more precisely a DAG,
and even sometimes a tree [scheduling has manufacturing roots]). This is the point of view of the parallel
computing community (see for example [SS05]), which is just a bit older than the real-time community: the one
emerged at the beginning of the sixties, the other at the end of the sixties, and during their first 40 years they
grew up with relatively few interactions. The second claim of this paper is that, even on our weakly-parallel
architectures (2 cores today, 4 cores tomorrow for critical systems; the many-core is another subject), taking
communications explicitly into account is essential to obtain realistic (not too optimistic) multi-core schedules.



bla.



bla.



bla.



desavantage de l'approche : on ne peut pas tricher et simplifier le probleme. Dommage !!!



\section{The Scade models}

bla.

% \begin{figure}[h]
% \includegraphics[width=1\textwidth]{scade_core_Ntasks}
% \caption{Fully dependent parallel tasks}
% \end{figure}

bla.

% \begin{figure}[h]
% \includegraphics[width=1\textwidth]{scade_Mcores_Ntasks}
% \caption{Fully dependent parallel tasks}
% \end{figure}

bla.



bla.

\section{Some results}

bla.

   1.5


\begin{tabular}{|c|c|}
\hline
MathSat 5.6.0  & 1.5s \\
\hline
Z3 4.8.4 & 4.3s \\
\hline
CVC4 1.6 & 15s \\
\hline
Yices 2.6.0 & 13mn \\
\hline
DV 6.6 & 45mn \\
\hline
\end{tabular}

Scade Prover is not yet able to generate the scheduling configuration,
we ignore if the problem is the CSP internally generated of the solver itself.

\section{Conclusions and Future Work}




bla.


shannon as a phase transition

*********

Pragmatic Information Rates, Generalizations of the
Kelly Criterion, and Financial Market Efficiency
Edward D Weinberger, Ph.D

https://arxiv.org/ftp/arxiv/papers/0903/0903.2243.pdf

In physics, phase transitions (sudden, qualitative changes in a physical system with small
changes in temperature, pressure, or other state variable) are characterized by a
discontinuity in the physical entropy of the system or its derivative with respect to that
state variable (Thompson, 1979). The Noisy Coding Theorem is effectively the statement
that the mutual information has a discontinuous derivative with respect to the input 
Pragmatic Information Rates, Generalizations of the Kelly Criterion, and … 20
entropy rate at the channel capacity. Since the input entropy rate to a communications
channel is the analogue of the physical entropy, the Noisy Coding Theorem is therefore
also the statement that there is a phase transition of the mutual information of the
channel. In fact, papers such as (Kabashima and Saad, 1999) study a class of error
correcting codes using the detailed formalism developed to describe physical phase
transitions.

Kabashima, A. and Saad, D. (1999). “Statistical mechanics of error correcting codes,”
Europhysics Letters, 45, No. 4, 97-103.


*******


youtube.com

Networks are everywhere with Albert-László Barabási

Introduction to Complexity: Scale-Free and Long-Tailed Degree Distributions Part 1

Remco van der Hofstad - The Structure of Complex Networks: Scale-Free and Small-World Random Graphs   40ième mn , 52eme : no phase transition

https://www.networkpages.nl/scale-free-networks-a-controversial-topic-properly-solved-by-extreme-mathematics/
réponse à "power-law is rare"

%%
%% Bibliography
%%

%% Please use bibtex, 

\bibliography{lectures}

\end{document}
